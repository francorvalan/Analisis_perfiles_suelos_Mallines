% Options for packages loaded elsewhere
\PassOptionsToPackage{unicode}{hyperref}
\PassOptionsToPackage{hyphens}{url}
%
\documentclass[
]{article}
\usepackage{amsmath,amssymb}
\usepackage{iftex}
\ifPDFTeX
  \usepackage[T1]{fontenc}
  \usepackage[utf8]{inputenc}
  \usepackage{textcomp} % provide euro and other symbols
\else % if luatex or xetex
  \usepackage{unicode-math} % this also loads fontspec
  \defaultfontfeatures{Scale=MatchLowercase}
  \defaultfontfeatures[\rmfamily]{Ligatures=TeX,Scale=1}
\fi
\usepackage{lmodern}
\ifPDFTeX\else
  % xetex/luatex font selection
\fi
% Use upquote if available, for straight quotes in verbatim environments
\IfFileExists{upquote.sty}{\usepackage{upquote}}{}
\IfFileExists{microtype.sty}{% use microtype if available
  \usepackage[]{microtype}
  \UseMicrotypeSet[protrusion]{basicmath} % disable protrusion for tt fonts
}{}
\makeatletter
\@ifundefined{KOMAClassName}{% if non-KOMA class
  \IfFileExists{parskip.sty}{%
    \usepackage{parskip}
  }{% else
    \setlength{\parindent}{0pt}
    \setlength{\parskip}{6pt plus 2pt minus 1pt}}
}{% if KOMA class
  \KOMAoptions{parskip=half}}
\makeatother
\usepackage{xcolor}
\usepackage[margin=1in]{geometry}
\usepackage{longtable,booktabs,array}
\usepackage{calc} % for calculating minipage widths
% Correct order of tables after \paragraph or \subparagraph
\usepackage{etoolbox}
\makeatletter
\patchcmd\longtable{\par}{\if@noskipsec\mbox{}\fi\par}{}{}
\makeatother
% Allow footnotes in longtable head/foot
\IfFileExists{footnotehyper.sty}{\usepackage{footnotehyper}}{\usepackage{footnote}}
\makesavenoteenv{longtable}
\usepackage{graphicx}
\makeatletter
\def\maxwidth{\ifdim\Gin@nat@width>\linewidth\linewidth\else\Gin@nat@width\fi}
\def\maxheight{\ifdim\Gin@nat@height>\textheight\textheight\else\Gin@nat@height\fi}
\makeatother
% Scale images if necessary, so that they will not overflow the page
% margins by default, and it is still possible to overwrite the defaults
% using explicit options in \includegraphics[width, height, ...]{}
\setkeys{Gin}{width=\maxwidth,height=\maxheight,keepaspectratio}
% Set default figure placement to htbp
\makeatletter
\def\fps@figure{htbp}
\makeatother
\setlength{\emergencystretch}{3em} % prevent overfull lines
\providecommand{\tightlist}{%
  \setlength{\itemsep}{0pt}\setlength{\parskip}{0pt}}
\setcounter{secnumdepth}{-\maxdimen} % remove section numbering
\usepackage{booktabs}
\usepackage{longtable}
\usepackage{array}
\usepackage{multirow}
\usepackage{wrapfig}
\usepackage{float}
\usepackage{colortbl}
\usepackage{pdflscape}
\usepackage{tabu}
\usepackage{threeparttable}
\usepackage{threeparttablex}
\usepackage[normalem]{ulem}
\usepackage{makecell}
\usepackage{xcolor}
\ifLuaTeX
  \usepackage{selnolig}  % disable illegal ligatures
\fi
\IfFileExists{bookmark.sty}{\usepackage{bookmark}}{\usepackage{hyperref}}
\IfFileExists{xurl.sty}{\usepackage{xurl}}{} % add URL line breaks if available
\urlstyle{same}
\hypersetup{
  pdftitle={Vegas Andinas análisis de datos Suelo},
  hidelinks,
  pdfcreator={LaTeX via pandoc}}

\title{Vegas Andinas análisis de datos Suelo}
\author{}
\date{\vspace{-2.5em}2024-04-26}

\begin{document}
\maketitle

\hypertarget{introducciuxf3n}{%
\section{1 Introducción}\label{introducciuxf3n}}

Se analizaron 51 localizados en 21, comprendiendo distintas
profundidades, abarcando desde la superfie hasta NA.

De los mismos, 6 perfiles presentaron problemas en sus profundidades,
los cuales se detallan a continuacion

\begin{longtable}[]{@{}cccccc@{}}
\caption{Clasificacion textural de las capas analizadas}\tabularnewline
\toprule\noalign{}
Mallin & top & bottom & SAND & SILT & CLAY \\
\midrule\noalign{}
\endfirsthead
\toprule\noalign{}
Mallin & top & bottom & SAND & SILT & CLAY \\
\midrule\noalign{}
\endhead
\bottomrule\noalign{}
\endlastfoot
M12 C1 & 0 & 30 & 48.5 & 34.7 & 16.8 \\
M12 C1 & 30 & agua & NA & NA & NA \\
M12 B & 0 & 20 & 39.8 & 42.1 & 18.1 \\
M12 B & 20 & agua & 50.9 & 25.1 & 24.0 \\
M33 & 0 & 35 & 71.0 & 23.9 & 5.1 \\
M33 & 35 & ROCA & NA & NA & NA \\
M36 & 0 & 43 & 43.6 & 41.1 & 15.3 \\
M36 & 43 & 93 & 88.0 & 11.0 & 1.0 \\
M36 & 93 & 117 & 83.4 & 16.3 & 0.3 \\
M36 & 117 & 232 & 45.5 & 47.1 & 7.5 \\
M36 & 232+ & + & 44.4 & 39.4 & 16.2 \\
M4.1 C1 & 0 & 20 & 51.5 & 33.0 & 15.5 \\
M4.1 C1 & 20 & 32 & 41.9 & 37.4 & 20.7 \\
M4.1 C1 & 32 & 70 & 9.5 & 39.1 & 51.4 \\
M4.1 C1 & 70 + & + & 50.7 & 32.2 & 17.1 \\
M44 A & 0 & 17 & 65.6 & 22.4 & 12.0 \\
M44 A & 17 & 55 & 60.7 & 25.7 & 13.6 \\
M44 A & 55 & L & NA & NA & NA \\
\end{longtable}

\hypertarget{graficos-de-perfiles}{%
\section{2 Graficos de perfiles}\label{graficos-de-perfiles}}

\includegraphics{Vegas_analisis_de_suelos_files/figure-latex/unnamed-chunk-6-1.pdf}
\includegraphics{Vegas_analisis_de_suelos_files/figure-latex/unnamed-chunk-6-2.pdf}
\includegraphics{Vegas_analisis_de_suelos_files/figure-latex/unnamed-chunk-6-3.pdf}
\includegraphics{Vegas_analisis_de_suelos_files/figure-latex/unnamed-chunk-6-4.pdf}
\includegraphics{Vegas_analisis_de_suelos_files/figure-latex/unnamed-chunk-6-5.pdf}
\includegraphics{Vegas_analisis_de_suelos_files/figure-latex/unnamed-chunk-6-6.pdf}
\includegraphics{Vegas_analisis_de_suelos_files/figure-latex/unnamed-chunk-6-7.pdf}
\includegraphics{Vegas_analisis_de_suelos_files/figure-latex/unnamed-chunk-6-8.pdf}
\includegraphics{Vegas_analisis_de_suelos_files/figure-latex/unnamed-chunk-6-9.pdf}
\includegraphics{Vegas_analisis_de_suelos_files/figure-latex/unnamed-chunk-6-10.pdf}
\includegraphics{Vegas_analisis_de_suelos_files/figure-latex/unnamed-chunk-6-11.pdf}
\includegraphics{Vegas_analisis_de_suelos_files/figure-latex/unnamed-chunk-6-12.pdf}
\includegraphics{Vegas_analisis_de_suelos_files/figure-latex/unnamed-chunk-6-13.pdf}
\includegraphics{Vegas_analisis_de_suelos_files/figure-latex/unnamed-chunk-6-14.pdf}
\includegraphics{Vegas_analisis_de_suelos_files/figure-latex/unnamed-chunk-6-15.pdf}

\hypertarget{anuxe1lisis-textural}{%
\section{3 Análisis Textural}\label{anuxe1lisis-textural}}

\hypertarget{clasificaciuxf3n-textural}{%
\subsection{3.1 Clasificación
textural}\label{clasificaciuxf3n-textural}}

A continuación se presenta los resultados de la clasificación textural
de los suelos de estudio utilizando el sistema de clasificación de USDA.

\begin{longtable}[]{@{}
  >{\centering\arraybackslash}p{(\columnwidth - 14\tabcolsep) * \real{0.1222}}
  >{\centering\arraybackslash}p{(\columnwidth - 14\tabcolsep) * \real{0.0556}}
  >{\centering\arraybackslash}p{(\columnwidth - 14\tabcolsep) * \real{0.0889}}
  >{\centering\arraybackslash}p{(\columnwidth - 14\tabcolsep) * \real{0.0667}}
  >{\centering\arraybackslash}p{(\columnwidth - 14\tabcolsep) * \real{0.0667}}
  >{\centering\arraybackslash}p{(\columnwidth - 14\tabcolsep) * \real{0.0667}}
  >{\centering\arraybackslash}p{(\columnwidth - 14\tabcolsep) * \real{0.2667}}
  >{\centering\arraybackslash}p{(\columnwidth - 14\tabcolsep) * \real{0.2667}}@{}}
\caption{Clasificación textural de las capas analizadas}\tabularnewline
\toprule\noalign{}
\begin{minipage}[b]{\linewidth}\centering
Mallin
\end{minipage} & \begin{minipage}[b]{\linewidth}\centering
top
\end{minipage} & \begin{minipage}[b]{\linewidth}\centering
bottom
\end{minipage} & \begin{minipage}[b]{\linewidth}\centering
SAND
\end{minipage} & \begin{minipage}[b]{\linewidth}\centering
SILT
\end{minipage} & \begin{minipage}[b]{\linewidth}\centering
CLAY
\end{minipage} & \begin{minipage}[b]{\linewidth}\centering
texture\_class
\end{minipage} & \begin{minipage}[b]{\linewidth}\centering
Clase\_textural
\end{minipage} \\
\midrule\noalign{}
\endfirsthead
\toprule\noalign{}
\begin{minipage}[b]{\linewidth}\centering
Mallin
\end{minipage} & \begin{minipage}[b]{\linewidth}\centering
top
\end{minipage} & \begin{minipage}[b]{\linewidth}\centering
bottom
\end{minipage} & \begin{minipage}[b]{\linewidth}\centering
SAND
\end{minipage} & \begin{minipage}[b]{\linewidth}\centering
SILT
\end{minipage} & \begin{minipage}[b]{\linewidth}\centering
CLAY
\end{minipage} & \begin{minipage}[b]{\linewidth}\centering
texture\_class
\end{minipage} & \begin{minipage}[b]{\linewidth}\centering
Clase\_textural
\end{minipage} \\
\midrule\noalign{}
\endhead
\bottomrule\noalign{}
\endlastfoot
Est2 A & 0 & 20 & 55.3 & 34.5 & 10.2 & Franco-arenoso & Franco
arenoso \\
Est2 A & 20 & 50 & 56.9 & 30.7 & 12.4 & Franco-arenoso & Franco
arenoso \\
Est2 B & 0 & 20 & 58.9 & 17.9 & 23.2 & Franco-arcillo-arenoso & Franco
arcillo arenoso \\
Est2 B & 20 & 40 & NA & NA & NA & & NA \\
M1 & 0 & 15 & 63.5 & 18.3 & 18.2 & Franco-arenoso & Franco arenoso \\
M14 A & 0 & 20 & 64.3 & 29.1 & 6.6 & Franco-arenoso & Franco arenoso \\
M14 A & 20 & 40 & 60.4 & 33.7 & 5.9 & Franco-arenoso & Franco arenoso \\
M14 A & 40 & 65 & NA & NA & NA & & NA \\
M14 C & 0 & 30 & 39.7 & 47.5 & 12.8 & Franco & Franco \\
M15 & 0 & 10 & 29.1 & 53.9 & 17.0 & Franco-limoso & Franco limoso \\
M15 & 10 & 30 & 10.5 & 86.1 & 3.4 & Franco-limoso & Limoso \\
M15 & 30 & 60 & 39.0 & 53.2 & 7.8 & Franco-limoso & Franco limoso \\
M15 & 60 & 80 & 78.3 & 17.4 & 4.3 & Arenoso-franco & Arenoso franco \\
M15 & 80 & 115 & 44.0 & 38.1 & 17.9 & Franco & Franco \\
M15 B & 0 & 20 & 71.7 & 25.0 & 3.3 & Arenoso-franco & Franco arenoso \\
M15 C & 0 & 30 & 32.9 & 52.7 & 14.4 & Franco-limoso & Franco limoso \\
M18 & 0 & 20 & 22.8 & 68.4 & 8.8 & Franco-limoso & Franco limoso \\
M18 (b) A & 0 & 40 & 41.4 & 25.9 & 32.7 & Franco-arcilloso & Franco
arcilloso \\
M18 (b) A & 40 & 55 & NA & NA & NA & & NA \\
M18(b) C & 0 & 60 & 79.7 & 14.5 & 5.8 & Arenoso-franco & Arenoso
franco \\
M19 & 0 & 22 & 6.9 & 82.1 & 11.0 & Franco-limoso & Limoso \\
M19 & 22 & 29 & 1.4 & 88.5 & 10.1 & Limoso & Limoso \\
M19 & 29 & 41 & 82.9 & 7.0 & 10.1 & Arenoso-franco & Arenoso franco \\
M19 & 41 & 64 & 21.8 & 62.0 & 16.2 & Franco-limoso & Franco limoso \\
M19 & 64 & 77 & 49.1 & 46.0 & 4.8 & Franco-arenoso & Franco arenoso \\
M19 & 77 & 140 & 31.6 & 44.2 & 24.2 & Franco & Franco \\
M2 A & 0 & 20 & 69.4 & 25.9 & 4.7 & Franco-arenoso & Franco arenoso \\
M2 A & 20 & 40 & 32.1 & 56.9 & 11.0 & Franco-limoso & Franco limoso \\
M2.2A C1 & 0 & 20 & 57.8 & 27.7 & 14.5 & Franco-arenoso & Franco
arenoso \\
M2.2A C1 & 20 & 32 & NA & NA & NA & & NA \\
M2.2A C1 & 32 & 70 & NA & NA & NA & & NA \\
M42 & 0 & 30 & 19.9 & 58.2 & 21.9 & Franco-limoso & Franco limoso \\
M42 & 30 & 50 & NA & NA & NA & & NA \\
\end{longtable}

\hypertarget{triuxe1ngulos-de-textura-usda}{%
\subsection{3.2 Triángulos de textura
(USDA)}\label{triuxe1ngulos-de-textura-usda}}

\includegraphics{Vegas_analisis_de_suelos_files/figure-latex/unnamed-chunk-11-1.pdf}

\includegraphics{Vegas_analisis_de_suelos_files/figure-latex/unnamed-chunk-12-1.pdf}

\includegraphics{Vegas_analisis_de_suelos_files/figure-latex/unnamed-chunk-13-1.pdf}

\end{document}
